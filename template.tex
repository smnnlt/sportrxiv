% Options for packages loaded elsewhere
\PassOptionsToPackage{unicode}{hyperref}
\PassOptionsToPackage{hyphens}{url}
\PassOptionsToPackage{dvipsnames,svgnames,x11names}{xcolor}
%
\documentclass[
]{article}

\usepackage{amsmath,amssymb}
\usepackage{lmodern}
\usepackage{iftex}
\ifPDFTeX
  \usepackage[T1]{fontenc}
  \usepackage[utf8]{inputenc}
  \usepackage{textcomp} % provide euro and other symbols
\else % if luatex or xetex
  \usepackage{unicode-math}
  \defaultfontfeatures{Scale=MatchLowercase}
  \defaultfontfeatures[\rmfamily]{Ligatures=TeX,Scale=1}
  \setmainfont[]{Open Sans}
\fi
% Use upquote if available, for straight quotes in verbatim environments
\IfFileExists{upquote.sty}{\usepackage{upquote}}{}
\IfFileExists{microtype.sty}{% use microtype if available
  \usepackage[]{microtype}
  \UseMicrotypeSet[protrusion]{basicmath} % disable protrusion for tt fonts
}{}
\makeatletter
\@ifundefined{KOMAClassName}{% if non-KOMA class
  \IfFileExists{parskip.sty}{%
    \usepackage{parskip}
  }{% else
    \setlength{\parindent}{0pt}
    \setlength{\parskip}{6pt plus 2pt minus 1pt}}
}{% if KOMA class
  \KOMAoptions{parskip=half}}
\makeatother
\usepackage{xcolor}
\setlength{\emergencystretch}{3em} % prevent overfull lines
\setcounter{secnumdepth}{-\maxdimen} % remove section numbering
% Make \paragraph and \subparagraph free-standing
\ifx\paragraph\undefined\else
  \let\oldparagraph\paragraph
  \renewcommand{\paragraph}[1]{\oldparagraph{#1}\mbox{}}
\fi
\ifx\subparagraph\undefined\else
  \let\oldsubparagraph\subparagraph
  \renewcommand{\subparagraph}[1]{\oldsubparagraph{#1}\mbox{}}
\fi


\providecommand{\tightlist}{%
  \setlength{\itemsep}{0pt}\setlength{\parskip}{0pt}}\usepackage{longtable,booktabs,array}
\usepackage{calc} % for calculating minipage widths
% Correct order of tables after \paragraph or \subparagraph
\usepackage{etoolbox}
\makeatletter
\patchcmd\longtable{\par}{\if@noskipsec\mbox{}\fi\par}{}{}
\makeatother
% Allow footnotes in longtable head/foot
\IfFileExists{footnotehyper.sty}{\usepackage{footnotehyper}}{\usepackage{footnote}}
\makesavenoteenv{longtable}
\usepackage{graphicx}
\makeatletter
\def\maxwidth{\ifdim\Gin@nat@width>\linewidth\linewidth\else\Gin@nat@width\fi}
\def\maxheight{\ifdim\Gin@nat@height>\textheight\textheight\else\Gin@nat@height\fi}
\makeatother
% Scale images if necessary, so that they will not overflow the page
% margins by default, and it is still possible to overwrite the defaults
% using explicit options in \includegraphics[width, height, ...]{}
\setkeys{Gin}{width=\maxwidth,height=\maxheight,keepaspectratio}
% Set default figure placement to htbp
\makeatletter
\def\fps@figure{htbp}
\makeatother
\newlength{\cslhangindent}
\setlength{\cslhangindent}{1.5em}
\newlength{\csllabelwidth}
\setlength{\csllabelwidth}{3em}
\newlength{\cslentryspacingunit} % times entry-spacing
\setlength{\cslentryspacingunit}{\parskip}
\newenvironment{CSLReferences}[2] % #1 hanging-ident, #2 entry spacing
 {% don't indent paragraphs
  \setlength{\parindent}{0pt}
  % turn on hanging indent if param 1 is 1
  \ifodd #1
  \let\oldpar\par
  \def\par{\hangindent=\cslhangindent\oldpar}
  \fi
  % set entry spacing
  \setlength{\parskip}{#2\cslentryspacingunit}
 }%
 {}
\usepackage{calc}
\newcommand{\CSLBlock}[1]{#1\hfill\break}
\newcommand{\CSLLeftMargin}[1]{\parbox[t]{\csllabelwidth}{#1}}
\newcommand{\CSLRightInline}[1]{\parbox[t]{\linewidth - \csllabelwidth}{#1}\break}
\newcommand{\CSLIndent}[1]{\hspace{\cslhangindent}#1}

\usepackage{orcidlink,fancyhdr,graphicx,xcolor,xstring,fontspec,mdframed}
\usepackage[margin=1in]{geometry}
\usepackage[onehalfspacing]{setspace}
\usepackage[scale=2.5]{ccicons}
\newcommand\shortauthor{}
\makeatletter
\makeatother
\makeatletter
\makeatother
\makeatletter
\@ifpackageloaded{caption}{}{\usepackage{caption}}
\AtBeginDocument{%
\ifdefined\contentsname
  \renewcommand*\contentsname{Table of contents}
\else
  \newcommand\contentsname{Table of contents}
\fi
\ifdefined\listfigurename
  \renewcommand*\listfigurename{List of Figures}
\else
  \newcommand\listfigurename{List of Figures}
\fi
\ifdefined\listtablename
  \renewcommand*\listtablename{List of Tables}
\else
  \newcommand\listtablename{List of Tables}
\fi
\ifdefined\figurename
  \renewcommand*\figurename{Figure}
\else
  \newcommand\figurename{Figure}
\fi
\ifdefined\tablename
  \renewcommand*\tablename{Table}
\else
  \newcommand\tablename{Table}
\fi
}
\@ifpackageloaded{float}{}{\usepackage{float}}
\floatstyle{ruled}
\@ifundefined{c@chapter}{\newfloat{codelisting}{h}{lop}}{\newfloat{codelisting}{h}{lop}[chapter]}
\floatname{codelisting}{Listing}
\newcommand*\listoflistings{\listof{codelisting}{List of Listings}}
\makeatother
\makeatletter
\@ifpackageloaded{caption}{}{\usepackage{caption}}
\@ifpackageloaded{subcaption}{}{\usepackage{subcaption}}
\makeatother
\makeatletter
\@ifpackageloaded{tcolorbox}{}{\usepackage[many]{tcolorbox}}
\makeatother
\makeatletter
\@ifundefined{shadecolor}{\definecolor{shadecolor}{rgb}{.97, .97, .97}}
\makeatother
\makeatletter
\makeatother
\ifLuaTeX
  \usepackage{selnolig}  % disable illegal ligatures
\fi
\IfFileExists{bookmark.sty}{\usepackage{bookmark}}{\usepackage{hyperref}}
\IfFileExists{xurl.sty}{\usepackage{xurl}}{} % add URL line breaks if available
\urlstyle{same} % disable monospaced font for URLs
\hypersetup{
  pdfauthor={John Doe; Second Person; Third Person},
  pdfkeywords={template, demo, exercise science},
  colorlinks=true,
  linkcolor={blue},
  filecolor={Maroon},
  citecolor={Blue},
  urlcolor={blue},
  pdfcreator={LaTeX via pandoc}}

\author{John Doe \and Second Person \and Third Person}
\date{}

\begin{document}

% set short author name
\newcounter{nauthor}
\stepcounter{nauthor}\stepcounter{nauthor}\stepcounter{nauthor}
\ifnum\thenauthor>2
\renewcommand{\shortauthor}{John Doe et al. }
\else 
\renewcommand{\shortauthor}{Doe \& Person \& Person }
\fi

\pagestyle{fancy}
\fancyhead[L]{\shortauthor(\the\year)}
\fancyhead[R]{}
\fancypagestyle{firstpage}{
  \fancyhf{}% clear default for head and foot
  \lhead{\href{https://sportrxiv.org}{\includegraphics[width = 40mm]{logo.png}}}
  \chead{PREPRINT - NOT PEER REVIEWED}
  \rhead{\ccby\vspace{0.2mm}}
  \lfoot{\color{gray}All authors have read and approved  this version  of the manuscript. \linebreak The manuscript was last updated on \today}
}
\begin{flushleft}
\begin{spacing}{2.5}
\thispagestyle{firstpage}
\vspace*{0.1cm}
{\huge{\textbf{Demo format to use as template for an article submitted
to the SportRxiv preprint server}}}
\end{spacing}
\vspace*{0.1cm}
John Doe\textsuperscript{1,2*}~\orcidlink{0000-0000-0000-0000}, Second
Person\textsuperscript{1}, Third Person\textsuperscript{2}
\\
\bigskip
\textsuperscript{1}One Affiliation\\ \textsuperscript{2}Another
Affiliation
\\
\bigskip
*Correspondence: 
\href{mailto:JD@example.org}{\color{black}JD@example.org}
\\
\bigskip
\begin{spacing}{1}
Cite as: \shortauthor(\the\year). Demo format to use as template for an
article submitted to the SportRxiv preprint server. \emph{SportRxiv.} \\
\end{spacing}
  \vspace{1mm} Supplementary Materials: \href{https://osf.io/}{https://osf.io/} \\
\vspace*{1cm}
\end{flushleft}


\newenvironment{abstractbox}
  {}

\surroundwithmdframed[
  backgroundcolor=gray!10,
  innertopmargin=10pt,
  innerbottommargin=10pt,
  linewidth=2pt,
  linecolor=gray!30,
  leftmargin=20pt,
  rightmargin=20pt,
  frametitle={\large{Abstract}},
]{abstractbox}

\begin{abstractbox}
  Place your abstract here. Use Markdown to style the abstract,
e.g.~write \texttt{*word*} to print a \emph{word} in italics. You can
use the Latex command \texttt{\textbackslash{}newline\textbackslash{}}
to create a linebreak in the abstract. \newline~\emph{Explanation}: The
second backslash ensures that styling can be applied directly after the
linebreak. This document is only a demo explaining how to use the
template. This document is only a demo explaining how to use the
template. This document is only a demo explaining how to use the
template. This document is only a demo explaining how to use the
template. This document is only a demo explaining how to use the
template. This document is only a demo explaining how to use the
template. This document is only a demo explaining how to use the
template. 
    \\ \\
  Keywords: \emph{template; demo; exercise science}
   
\end{abstractbox}\ifdefined\Shaded\renewenvironment{Shaded}{\begin{tcolorbox}[borderline west={3pt}{0pt}{shadecolor}, enhanced, frame hidden, breakable, sharp corners, boxrule=0pt, interior hidden]}{\end{tcolorbox}}\fi

\hypertarget{introduction}{%
\section{Introduction}\label{introduction}}

\textcolor{SRXIVgreen}{This is a template for submissions} to the
\href{https://sportrxiv.org/index.php/server}{SportRxiv} pre-print
server. It uses \href{https://quarto.org/}{Quarto} to generate a PDF
file in the desired style.

To get this template set up on your device, follow the instructions on
\href{https://github.com/smnnlt/sportrxiv/}{GitHub}.

Once everything is in place, you can start by editing the .qmd file.
First, replace the \texttt{articletitle}, \texttt{author} and
\texttt{affiliation} info in the YAML header at the top of the document.
Render the document in an appropriate GUI (e.g.~R Studio, Jupyter Lab,
VS Studio) or via the command line to see the resulting PDF file. Then
modify the \texttt{abstract}, \texttt{keywords} and optionally other
variables of the YAML header.

To write the main text of your article you can use
\href{https://www.markdownguide.org/basic-syntax/}{Markdown syntax}. You
can use \emph{italics},
\href{https://www.urbandictionary.com/define.php?term=Link}{links}, or

\begin{itemize}
\tightlist
\item
  Lists
\item
  Either
\item
  ordered
\item
  or unordered
\end{itemize}

\hypertarget{methods}{%
\section{Methods}\label{methods}}

\hypertarget{citations-and-references}{%
\subsection{Citations and References}\label{citations-and-references}}

Use the \texttt{bibliography.bib} file as your reference library. You
can cite references in the text
(\protect\hyperlink{ref-allaire2022}{Allaire et al., 2022}). The easiest
way to streamline this process, is by using the Visual Editor for Quarto
in R Studio in combination with Zotero. This allows you to insert
citations from your Zotero data base, while simultaneously updating your
.bib file.

\hypertarget{figures-and-tables}{%
\subsection{Figures and Tables}\label{figures-and-tables}}

You can include Figures and Tables via the Markdown syntax. A different
approach is to include/generate these via code chunks.

\hypertarget{code-chunks}{%
\subsection{Code Chunks}\label{code-chunks}}

You can insert Quarto code chunk to render arbitrary code (e.g.~R,
Python, Julia, \ldots). By default, only the output of code-chunks is
displayed, but you can override this with the local code-chunk options.

\hypertarget{math}{%
\subsection{Math}\label{math}}

You can display math using LaTex either inline \(x = \sigma²\) or in
blocks:

\begin{equation}\protect\hypertarget{eq-norm}{}{
f(x) = {{1}\over{\sigma}\sqrt{2\pi}}e^{-{1\over{2}}({x-\mu\over{\sigma}})^2}
}\label{eq-norm}\end{equation}

Insert an equation label at the end of the equation to create the
equation number on the right side.

\hypertarget{cross-references}{%
\subsection{Cross References}\label{cross-references}}

You can create cross references to images, tables and equations. You
first need to insert an reference label (e.g.\{\#eq-norm\}) next to the
part you want to reference to. Then you can link to it in your Markdown
text, for example Equation~\ref{eq-norm}.

\hypertarget{results}{%
\section{Results}\label{results}}

You may change the general structure of the document by creating new or
modifying existing (sub)headings.

\hypertarget{discussion}{%
\section{Discussion}\label{discussion}}

Place your discussion here

\hypertarget{summary}{%
\section{Summary}\label{summary}}

Here is my summary.

\hypertarget{contributions}{%
\section{Contributions}\label{contributions}}

Authors should report the contributions of each author in the a specific
contribution section based on the guidelines set forth by the
International Committee of Medical Journal Editors.

Please indicate author contributions as clearly as possible, according
to the following criteria:

\begin{itemize}
\tightlist
\item
  Substantial contributions to conception and design
\item
  Acquisition of data
\item
  Analysis and interpretation of data
\item
  Drafting the article or revising it critically for important
  intellectual content
\item
  Final approval of the version to be published
\end{itemize}

\hypertarget{acknowledgements}{%
\section{Acknowledgements}\label{acknowledgements}}

People who contributed to the work but do not fit our author criteria
should be listed in the acknowledgements, along with their
contributions. You must ensure that anyone named in the acknowledgements
agrees to being so named.

Funding sources should not be included in the acknowledgements, but in
the section below.

\hypertarget{funding-information}{%
\section{Funding Information}\label{funding-information}}

Please provide a list of the sources of funding, as well as the relevant
grant numbers, where possible. List the authors associated with specific
funding sources. You will also enter this information in a form during
the submission process, but it must be repeated here.

\hypertarget{data-and-supplementary-material-accessibility}{%
\section{Data and Supplementary Material
Accessibility}\label{data-and-supplementary-material-accessibility}}

This should list the database(s) and, if appropriate, the respective
accession numbers and DOIs for all data or supplementary material for
the manuscript that has been made publicly available on a trusted
digital repository. If no data, code, or supplementary material are
available for this manuscript then the reason for this should be
explained here.

\hypertarget{bibliography}{%
\section*{References}\label{bibliography}}
\addcontentsline{toc}{section}{References}

\hypertarget{refs}{}
\begin{CSLReferences}{1}{0}
\leavevmode\vadjust pre{\hypertarget{ref-allaire2022}{}}%
Allaire, J. J., Teague, C., Scheidegger, C., Xie, Y., \& Dervieux, C.
(2022). \emph{Quarto}. \url{https://doi.org/10.5281/zenodo.5960048}

\end{CSLReferences}



\end{document}
